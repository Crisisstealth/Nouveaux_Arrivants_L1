\documentclass[]{report}

\usepackage{color}

\definecolor{dkgreen}{rgb}{0,0.6,0}
\definecolor{gray}{rgb}{0.5,0.5,0.5}
\definecolor{mauve}{rgb}{0.58,0,0.82}

\usepackage[utf8]{inputenc}
\usepackage[T1]{fontenc}
\usepackage[light, largesmallcaps, oldstyle]{kpfonts}
\usepackage[french]{babel}
\usepackage{listings}
\usepackage{titlesec}
\usepackage{lipsum}
\usepackage{charter}
\usepackage{graphicx}

\titleformat{\chapter}[display]
  {\normalfont\bfseries}{}{0pt}{\Large}
 
\titleformat{\section}[display]
  {\normalfont\bfseries}{}{0pt}{\large}
  
\titleformat{\subsection}[display]
  {\normalfont\bfseries}{}{0pt}{\normalsize}
  
 \lstset{frame=tb,
  language=C,
  aboveskip=3mm,
  belowskip=3mm,
  showstringspaces=false,
  columns=flexible,
  basicstyle={\small\ttfamily},
  numbers=none,
  numberstyle=\tiny\color{gray},
  keywordstyle=\color{blue},
  commentstyle=\color{dkgreen},
  stringstyle=\color{mauve},
  breaklines=true,
  breakatwhitespace=true,
  tabsize=4
}

\usepackage{geometry}
\geometry{top=2cm, bottom=2cm, left=2cm, right=2cm}


\begin{document}
\begin{titlepage}
   \begin{center}
       \vspace*{1cm}

       \Large{\textbf{Fiche d'introduction aux UE des cycles préparatoires}}

       \vspace{0.5cm}
       Avis et méthodes des Étudiants de la Faculté des Sciences et Ingénierie, groupe Sorbonne Universités
            
       \vspace{1.5cm}

       \textbf{L'équipe des Étudiants pilotes}
       \medskip \\
       Document à destination des nouveaux arrivants \\ sur le campus Pierre et Marie Curie
            
       \vspace{0.8cm}
     
       \includegraphics[width=0.4\textwidth]{/home/theo/Images/Marie_Curie_Radium.jpeg} \\
       \bigskip
       
       \large{\textbf{Avec la collaboration de :}\\ \normalsize{Nom des collaborateurs.}}
            
   \end{center}
\end{titlepage}


\chapter{\centerline{Introduction}}

\textit{Pour commencer, Félicitations et Bienvenue à la Faculté des Sciences et Ingénierie du groupe Sorbonne Universités. Nous espérons que vous apprécierez votre parcours parmi nous et que vous en ressortirez plus que satisfaits. Cette fiche est le fruit de la collaboration de plusieurs étudiants en cycle préparatoire (L1) issus de différentes sections et de différents niveaux afin de vous donner la vision la plus générale et objective possible sur cette année. Cette fiche traitera des différentes UE (Unités d'enseignement) et de la méthode de travail à adopter. \medskip}

\textit{{\emph{Attention : la méthode de travail varie entre chaque étudiant, cette fiche n'est donnée qu'à titre informatif et ne doit en aucun cas être lue comme une méthode à suivre en toute situation. Ce sera l'expérience et votre manière d'aborder les UE qui dicteront votre méthode, cette fiche sert à vous donner des pistes à explorer afin de vous aider à trouver \emph{votre} méthode.}}}
\newpage

\tableofcontents
\chapter{\centerline{Conseils généraux}}

\chapter{\centerline{Premier Semestre}}
\section{UE communes}
\textit{Bien qu'il existe différents parcours de cycle d'intégration, ces derniers comprennent des UE qui leurs sont communes, et à ne pas négliger!}

\subsection{\quad \underline{Orientation et Insertion Professionnelle (3 ECTS)}}
C'est une UE obligatoire qui vous aidera à faire vos premiers pas dans le monde Universitaire. Elle vous apprendra à vous exprimer à l'oral, mais aussi à rédiger des documents et créer des powerpoint dans une syntaxe et une mise en page parfaites qui seront attendues dans le monde professionnel ou encore à vous initier à la recherche documentaire. Vous aurez un projet à rendre en fin de semestre avec un oral, le tout en groupe.\smallskip \\
C'est une UE qui rapporte des points faciles, donc prenez là avec un minimum de sérieux même si elle peut vous paraître obsolète. La présence est obligatoire et les travaux à rendre prennent un temps assez conséquent, donc prévoyez ces derniers.
\subsection{ \quad \underline{Mathématiques pour les Sciences 1 (9 ECTS)}}

\section{UE partagées}
Au premier Semestre, vous suivrez trois des UE qui vous sont proposées ci-dessous \emph{en fonction de votre section.} Par exemple, un étudiant en BGC ne peut pas s'inscrire en électronique ou Physique (Optique et Électrocinétique), bien que la Mécanique-Physique reste accessible, comme un MIPI ne peut accéder à la Biologie ou la Chimie. En bref, faites preuve de discernement pendant vos choix!
\subsection{\quad \underline{Chimie: structure et réactivité (6 ECTS)}}
Ici, on apprend les bases de l'atomistique en suivant trois grandes lignes: Atomes, Molécules et Réactivité. C'est une UE qui peut paraître assez abstraite au début car elle pourra donner tort à ce que vous avez vu au lycée -- sans pour autant que ça ait été faux à l'instant où vous l'avez appris, mais parce que c'est un domaine de la chimie qui évolue encore.Par exemple, on vous dira très vite d'arrêter de penser que les électrons tournent en jolis cercles bien concentriques autour du noyau, ce n'est pas du tout le cas.\smallskip \\ 
Il est cependant absolument \textbf{indispensable} de traiter ces trois parties comme un tout et non à part, en effet, c'est une UE très logique, et avec peu de calcul numérique! Elle se base surtout sur une bonne compréhension des atomes en fonction de leur place dans le tableau périodique, une bonne capacité de modélisation en 3D et surtout une capacité à faire le lien entre les différents chapitres. Cette dernière compétence vous sera très utile, car tous les chapitres sont étroitement liés : vous ne pouvez pas expliquer pourquoi $CH_4$ compte exactement quatre liaisons simples sans connaître les propriétés de chaque atome (je vous rassure tout de suite, ce sera bien plus profond et complexe que ça). \newpage
Mais alors comment la bosser? Rien de plus facile ! \emph{Allez en TD et révisez fréquemment.} Comme je vous l'ai dit, c'est une UE très logique et qui ne nécessite pas tant de travail que ça pour obtenir une très bonne note. Les amphis vous seront très utiles à la compréhension et abordent des cas concrets, et les TD servent à renforcer ces connaissances par l'application.
\subsection{\quad \underline{Introduction à l'électronique (6 ECTS)}}
\subsection{\quad \underline{Éléments de programmation 1 (6 ECTS)}}
\subsection{\quad \underline{Optique et Électrocinétique (6 ECTS)}}
\subsection{\quad \underline{Mécanique Physique 1 (6 ECTS)}}
\subsection{\quad \underline{Organisations cellulaires du vivant (6 ECTS)}}
\subsection{\quad \underline{Géosciences 1: Le système Terre (6 ECTS)}}

\chapter{\centerline{Deuxième Semestre}}
\section{UE communes}
\subsection{\quad \underline{Mathématiques pour les sciences 2 (6 ECTS)}}
\subsection{\quad \underline{Initiation aux mineures transdisciplinaires thématiques (IMTT) \emph{ou} Atelier de Recherche Encadré (ARE) (3 ECTS)}}

\subsubsection{\quad \quad ARE : ChemCraft}
\subsubsection{\quad \quad ARE : }

\subsection{\quad \underline{Anglais L1 (3 ECTS)}}

\section{UE partagées}
Au second semestre, votre champ de disciplines se réduit quelque peu, vous ne pourrez choisir, à votre grand désarroi, que deux des UE ci-dessous. Le principe d'exclusivité décrit pour le premier semestre est toujours en vigueur.
\subsection{\quad \underline{Transformations chimiques en solution aqueuse (9 ECTS)}}
\subsection{\quad \underline{Fondements de l'électronique (9 ECTS)}}
\subsection{\quad \underline{Éléments de programmation 2 (9 ECTS)}}
\subsection{\quad \underline{Mathématiques approfondies (9 ECTS)}}
\subsection{\quad \underline{Mécanique Physique 2 (9 ECTS) }}
\subsection{\quad \underline{Organisation moléculaire du vivant (9 ECTS)}}
\subsection{\quad \underline{Géosciences 2 (9 ECTS)}}
\subsection{\quad \underline{Projet FABLAB}}

\chapter{\centerline{Pour aller plus loin}}
\end{document}
